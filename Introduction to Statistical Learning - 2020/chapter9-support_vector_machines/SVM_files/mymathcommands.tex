%S�tze, Definitionen , Beispiele und Beweise

%\newcounter{claim}[section]
%\addtocounter{claim}{0} 


%\theoremstyle{plain}
%\newtheorem*{theo}{Theorem \thesection.\arabic{claim}} 
%\newtheorem*{lem}{Lemma \thesection.\arabic{claim}}
%\newtheorem*{cor}{Corollary \thesection.\arabic{claim}}
%\newtheorem*{propo}{Proposition \thesection.\arabic{claim}}

%\theoremstyle{definition}
%\newtheorem*{define}{Definition \thesection.\arabic{claim}}

%\theoremstyle{remark}
%\newtheorem*{bem}{Remark \thesection.\arabic{claim}}
%\newtheorem*{bsp}{Example \thesection.\arabic{claim}}

%\newcommand{\definition}[1]{\stepcounter{claim} \begin{define}#1\end{define}}
%\newcommand{\remark}[1]{\stepcounter{claim} \begin{bem}#1\end{bem}}
%\newcommand{\example}[1]{\stepcounter{claim} \begin{bsp}#1\end{bsp}}
%\newcommand{\pf}[1]{\begin{proof}#1\end{proof}}
%\newcommand{\theorem}[1]{\stepcounter{claim} \begin{theo}#1\end{theo}}
%\newcommand{\proposition}[1]{\stepcounter{claim} \begin{propo}#1\end{propo}}
%\newcommand{\lemma}[1]{\stepcounter{claim} \begin{lem}#1\end{lem}}
%\newcommand{\corollary}[1]{\stepcounter{claim} \begin{cor}#1\end{cor}}


\theoremstyle{plain}
\newtheorem{theo}{Theorem} \numberwithin{theo}{section} 
\newtheorem{lem}[theo]{Lemma}
\newtheorem{cor}[theo]{Corollary}
\newtheorem{propo}[theo]{Proposition}
\newtheorem{problem}[theo]{Problem}

\theoremstyle{definition}
\newtheorem{define}[theo]{Definition}

\theoremstyle{remark}
\newtheorem{bem}[theo]{Remark}
\newtheorem{bsp}[theo]{Example}


\newcommand{\definition}[1]{\begin{define}#1\end{define}}
\newcommand{\remark}[1]{\begin{bem}#1\end{bem}}
\newcommand{\example}[1]{\begin{bsp}#1\end{bsp}}
\newcommand{\pf}[1]{\begin{proof}#1\end{proof}}
\newcommand{\theorem}[1]{\begin{theo}#1\end{theo}}
\newcommand{\proposition}[1]{\begin{propo}#1\end{propo}}
\newcommand{\lemma}[1]{\begin{lem}#1\end{lem}}
\newcommand{\corollary}[1]{\begin{cor}#1\end{cor}}
\newcommand{\Problem}[1]{\begin{problem}#1\end{problem}}

% Referenzen
\newcommand{\theoref}{\ref}


% Aufz�hlungen
\newcommand{\stichpunkte}[1]{
\begin{itemize}
	#1
\end{itemize}
}

\newcommand{\enumeration}[1]{
\begin{enumerate}
	#1
\end{enumerate}
}

\newcommand{\case}[1]{
\begin{cases}
#1
\end{cases}
}

% Matrizen
\newcommand{\matrice}[1]{\begin{pmatrix}#1\end{pmatrix}}

%Gleichungen
\newcommand{\eqns}[1]{\begin{eqnarray*}#1\end{eqnarray*}}
\newcommand{\eqn}[1]{\begin{equation}#1\end{equation}}
\newcommand{\eqnumformung}[1]{\begin{equation*}
\begin{array}{rccl}#1
\end{array}
\end{equation*}
}

% Mengen

\newcommand{\M}[1]{\mathcal{#1}} % f�r Mengen 
\newcommand{\Hi}{\M{H}}          % Hilbertraum



\newcommand{\C}{\mathbb{C}}   %komplexe Zahlen
\newcommand{\R}{\mathbb{R}}   %relle Zahlen
\newcommand{\Q}{\mathbb{Q}}   %rationale Zahlen
\newcommand{\Z}{\mathbb{Z}}   %ganze Zahlen
\newcommand{\N}{\mathbb{N}}   %nat�rliche Zahlen
\newcommand{\K}{\mathbb{K}}   %K�rper
\newcommand{\T}{\mathbb{T}}   %Torus



\newcommand{\SL}{\mathcal{L}} %script L
\newcommand{\SM}{\mathcal{M}} %script M
\newcommand{\SH}{\mathcal{H}} %script H
\newcommand{\SK}{\mathcal{K}} %script K
\newcommand{\SA}{\mathcal{A}} %script A
\newcommand{\SB}{\mathcal{B}} %script B
\newcommand{\SC}{\mathcal{C}} %script C
\newcommand{\SD}{\mathcal{D}} %script D
\newcommand{\SF}{\mathcal{F}} %script F
\newcommand{\SO}{\mathcal{O}} %script O
\newcommand{\SR}{\mathcal{R}} %script R
\newcommand{\sS}{\mathcal{S}} %script S
\newcommand{\SU}{\mathcal{U}} %script U
\newcommand{\SV}{\mathcal{V}} %script V
\newcommand{\SW}{\mathcal{W}} %script W

\newcommand{\sg}{\mathfrak{g}}   %Lie Algebra g
\newcommand{\sh}{\mathfrak{h}}   %Lie Algebra h
\newcommand{\so}{\mathfrak{o}}   %Lie Algebra o
\newcommand{\su}{\mathfrak{u}}   %Lie Algebra u
\newcommand{\sgl}{\mathfrak{gl}} %Lie Algebra gl
\newcommand{\spu}{\mathfrak{pu}} %Lie Algebra pu


% Funktoren
\newcommand{\fL}{\textbf{L}} % Lie-Funktor L

% Konstanten
\DeclareMathOperator*{\I}{i}     %% imagin�re Einheit
\DeclareMathOperator*{\E}{e}     %% Euler�sche Zahl

% griechische Kleinbuchstaben

\newcommand{\eps}{\varepsilon}

% Differentialoperatoren

\newcommand{\mydiv}{\text{ div}}
\newcommand{\pdt}{\frac{\partial}{\partial t}}
\newcommand{\pd}[2]{\frac{\partial #1}{\partial #2}}
\newcommand{\dt}{\frac{d}{dt}}

% Mathe-Operatoren
\DeclareMathOperator*{\Tr}{Tr}     %% Spur; soll in S�tzen immer normal gedruckt erscheinen (nicht kursiv)
\DeclareMathOperator*{\im}{im}     %% Bild; soll in S�tzen immer normal gedruckt erscheinen (nicht kursiv)
\DeclareMathOperator*{\Ker}{ker}   %% Kern; soll in S�tzen immer normal gedruckt erscheinen (nicht kursiv)
\DeclareMathOperator*{\lin}{lin}   %% lineare H�lle; soll in S�tzen immer normal gedruckt erscheinen (nicht kursiv)
\DeclareMathOperator*{\Hom}{Hom}   %% Homomorphismen; soll in S�tzen immer normal gedruckt erscheinen (nicht kursiv)
\DeclareMathOperator*{\End}{End}   %% Endomorphismen; soll in S�tzen immer normal gedruckt erscheinen (nicht kursiv)
\DeclareMathOperator*{\Aut}{Aut}   %% Automorphismen; soll in S�tzen immer normal gedruckt erscheinen (nicht kursiv)
\DeclareMathOperator*{\Iso}{Iso}   %% Isomorphismen; soll in S�tzen immer normal gedruckt erscheinen (nicht kursiv)
\DeclareMathOperator*{\Mot}{Mot}   %% Bewegungsgruppe; soll in S�tzen immer normal gedruckt erscheinen (nicht kursiv)
\DeclareMathOperator*{\Aff}{Aff}   %% Affine Gruppe; soll in S�tzen immer normal gedruckt erscheinen (nicht kursiv)
\DeclareMathOperator*{\Diff}{Diff} %% Diffeomorphismen; soll in S�tzen immer normal gedruckt erscheinen (nicht kursiv)
\DeclareMathOperator*{\GL}{GL}     %% invertierbare Operatoren; soll in S�tzen immer normal gedruckt erscheinen (nicht kursiv)
\DeclareMathOperator*{\Mat}{Mat}   %% Matrizen; soll in S�tzen immer normal gedruckt erscheinen (nicht kursiv)
\DeclareMathOperator*{\diag}{diag} %% Diagonalmatrizen; soll in S�tzen immer normal gedruckt erscheinen (nicht kursiv)
\DeclareMathOperator*{\Ad}{Ad}     %% Adjungierte auf Gruppenebene; soll in S�tzen immer normal gedruckt erscheinen (nicht kursiv)
\DeclareMathOperator*{\ad}{ad}     %% Adjungierte auf Algebrenebene; soll in S�tzen immer normal gedruckt erscheinen (nicht kursiv)
\DeclareMathOperator*{\Der}{Der}   %% Derivationen; soll in S�tzen immer normal gedruckt erscheinen (nicht kursiv)
\DeclareMathOperator*{\diam}{diam} %% Durchmesser; soll in S�tzen immer normal gedruckt erscheinen (nicht kursiv)
\DeclareMathOperator*{\id}{id}     %% Identit�t; soll in S�tzen immer normal gedruckt erscheinen (nicht kursiv)
\DeclareMathOperator*{\supp}{supp} %% Tr"ager einer Funktion; soll in S�tzen immer normal gedruckt erscheinen (nicht kursiv)
\DeclareMathOperator*{\Pro}{P}     %% Projektiv-Funktor; soll in S�tzen immer normal gedruckt erscheinen (nicht kursiv)

%Normen, Skalarprodukte, Erzeugnisse
\newcommand{\mynorm}[2]{\left\|#1\right\|_{#2}}
\newcommand{\opnorm}[1]{\mynorm{#1}{\text{op}}}
\newcommand{\skp}[3]{\left\langle #1,#2 \right\rangle_{#3}}
\newcommand{\gen}[2]{\left\langle #1\right\rangle_{#2}}


%Style
\newcommand{\oline}[1]{\overline{#1}}
\newcommand{\iof}{if and only if }                           %% if and only if
%\newcommand{\im}{\text{im}}                                   %% prints im(...) in math mode
%\newcommand{\lin}[1]{\text{span}\left\{#1\right\}}           %% prints span{...} in math mode
%\newcommand{\Hom}[1]{\text{Hom}\left(#1\right)}              %% prints Hom(...) in math mode 
%\newcommand{\Iso}[1]{\text{Iso}\left(#1\right)}              %% prints Iso(...) in math mode 
\newcommand{\1}{\mathbbm{1}}                                 %% prints the identity operator in math mode 

%Sonstiges

\newcommand{\mybinom}[2]{\left(^{#1}_{#2}\right)}  %% Binomialkoeffizienten